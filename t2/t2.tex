\documentclass[a4paper,12pt]{article}
\usepackage[utf8]{inputenc}
\usepackage[ngerman]{babel}
\usepackage[T1]{fontenc}
\usepackage[default]{raleway}
\usepackage{mathtools}
\usepackage[dvipsnames]{xcolor}
\usepackage[marginal, norule, perpage]{footmisc}

\renewcommand{\thefootnote}{\Roman{footnote}}

\usepackage{hyperref}
\hypersetup{
  colorlinks=true,
  linkcolor=MidnightBlue,
  urlcolor=MidnightBlue
}

%opening
\title{Physik Zusammenfassung\\	\large{Jahrgang 4 - Semester 1 - Test 2}}
\author{Markus Reichl}

\begin{document}

\maketitle

\tableofcontents

\newpage

\section{Wärmeübertragung}
\subsection{Arten der Wärmeübertragung}
\subsubsection{Wärmeleitung}
Wärmetransport durch Zusammenstoß (Bewegung) der Moleküle.
\subsubsection{Wärmeströmung}
Erfolgt in Flüssigkeiten und Gasen. Der Wärmetransport tritt in Zusammenhang mit einem Messetransport auf. (Bsp.: Zentralheizung)
\subsubsection{Wärmestrahlung}
Ausbreitung der Wärmeenergie von der Sonne zur Erde. (Der menschliche Körper strahl ca. 100 Watt ab)
\subsubsection{Bsp.: Thermoskanne}
Die Hülle der Kanne verringert den Luftdruck, dadurch werden die Wärmeleitung und die Wärmeströmung behindert.
Die Innenseite der Hülle ist verspiegelt, wodurch die Wärmestrahlung verhindert wird.

\subsection{Wärmeleitzahl $\lambda$}
Die Wärmeleitzahl $\lambda$ gibt an welcher Wärmestrom in Watt ständig durch eine 1$m^2$ große und 1m dicke Schicht eines Stoffes hindurchgeht, wenn $\Delta$T 1C$^\circ$ beträgt.

\subsection{Wärmedurchlasswiderstand D}
Der Wärmedurchlasswiderstand D ist in $\frac{m^2 K}{W}$ angegeben und wie folgt definiert:
$$D = \frac{d}{\lambda}$$
$$D = \sum \frac{d_i}{\lambda_i}$$
d ... Dichte\\
$\lambda$ ... Wärmeleitzahl

\subsection{Wärmedurchgangskoeffizient U}
Der Wärmedurchgangskoeffizient U ist in $\frac{W}{m^2 K}$ angegeben und wie folgt definiert:
$$U = \frac{1}{\frac{1}{\alpha_a} + \frac{1}{\alpha_i} + D}$$
$\frac{1}{\alpha}$ ... Übergangswiderstand\\
D ... Wärmedurchlasswiderstand

\subsection{Wärmeausdehnung}
\subsubsection{Bsp.: Wärmeausdehnung am Beispiel Wand}
Angegeben sind diverse Eigenschaften eines Objekts und gesucht wird der Wärmedurchgangskoeffizient.

Die Wand hat einen Außenputz von 1.5cm und eine Wärmeleitzahl $\lambda = 0.8 \frac{W}{m K}$. Der Verputz stellt einen Widerstand von $\alpha = 0.015m$ dar, welcher in die Formel für den Durchlasswiderstand D eingesetzt werden kann.
$$\frac{\alpha}{\lambda} = D = \frac{0.015}{0.8} \approx 0.018\frac{m^2 K}{W}$$
Die Wand besteht aus Hochziegeln mit $\lambda = 0.33\frac{W}{m K}$ und einer Dicke von 38cm. Hierzu dieselbe Rechnung.
$$D = \frac{0.38}{0.33} \approx 1.15\frac{m^2 K}{W}$$

Die Innenwand entspricht vom Aufbau der Außenwand daher gilt für diese auch $D \approx 0.018\frac{m^2 K}{W}$.

Der Gesamtwärmewiderstand entspricht der Summe aller Teilwiderstände.
$$D = \sum D_i = 1.21\frac{m^2 K}{W}$$

Damit kann in die Formel für den Wärmedurchgangskoeffizienten eingesetzt werden.
$$U = \frac{1}{\frac{1}{\alpha_a} + \frac{1}{\alpha_i} + D}$$
$$U = \frac{1}{0.18 + 1.21} \approx 0.73\frac{W}{m^2 K}$$

\subsubsection{Bsp.: Wärmeausdehnung am Beispiel Haus}
Das gleiche Prinzip wie im vorangegangenen Abschnitt kann auch auf größe Objekte angewandt werden.

Als Beispiel liegt ein Haus mit einer Grundfläche von 132$m^2$ vor. Die Außenwände haben eine Gesamtfläche von 128$m^2$ und einen Wärmedurchgangskoeffizienten von 0.7$\frac{W}{m^2 K}$. 
Die Decke, der Fußboden, sowie der Keller sind unbeheizt und haben jeweils einen Wärmedurchgangskoeffizienten von $1.1\frac{W}{m^2 K}$ und einen Faktor von 0.5 zusätzlich. 
Die Fenster haben extra eine Fläche von 22.7$m^2$ und einen Wärmedurchgangskoeffizienten von $2.2$. 

Es soll ermittelt werden wie viel Wärmeenergie in $\frac{W}{K}$ im Haus verloren geht, was durch die Fläche multipliziert mit dem Wärmedurchgangskoeffizienten beschrieben werden kann. 
Häufig ist noch ein weiterer Faktor angegeben welcher miteinberechnet wird.

Der Verlust durch die einzelnen Teile des Hauses werden also mit der Formel $A * U * f$ angegeben. Der Gesamtverlust entspricht der Summe dieser Teile.
$$Decke = 132 * 1.1 \approx 145.2$$
$$Fussboden = 132 * 1.1 * 0.5 \approx 72.6$$
$$Aussenwand = 128 * 0.7 \approx 93.4$$
$$Fenster = 22.7 * 2.2 \approx 49.9$$
$$Haus = Decke + Fussboden + Aussenwand + Fenster$$
$$Haus = 145.2 + 72.6 + 93.4 + 49.9 = 361.1\frac{W}{K}$$

\subsection{Normbemessungstemperaturen}
Normbemessungstemperaturen sind Temperaturen welche immer eindeutig bestimmt sind. Zu diesen zählt zum Beispiel die Raumtemperatur mit 20$^\circ$ Grad.

\subsubsection{Bsp.: Heizleistung}
Die Heizleistung kann durch die Leistung in Watt pro Kelvin multipliziert mit der Temperaturänderung angegeben werden.
Als Beispiel nehme man eine Außentemperatur von -18$^\circ$ Grad Celsius. Diese ist 38 Kelvin von der Raumtemperatur entfernt d. h. $\Delta$T = 38K.
Die Heizleistung bei einer Leistung von 3611$\frac{W}{K}$ beträgt damit 13.7kW.

\subsection{Wärmeübertragung durch Strahlung}
Körper strahlen abhängig von ihrer Temperatur, sowie ihrer Wellenlänge unterschiedlich viel Energie aus.
\subsubsection{Wien'sches Verschiebungsgesetz}
$$\lambda_{max} * T = k \text{in Km}$$
\subsubsection{Bsp.: Sonnenstrahlung}
Das Maximale Emissionsvermögen der Sonne bei einer Wellenlänge von 480nm mit einer Konstanten k von $2.9 * 10^{-3}Km$ soll mit den Wien'schen Verschiebungsgesetz berechnet werden.
Dafür muss die Wellenlänge in Meter umgerechnet werden, wobei man auf $48 * 10^{-8}$ kommt.
$$\lambda_{max} = \frac{k}{T} = \frac{2.9 * 10^{-3}}{48 * 10^{-8}} = \frac{29 * 10^4}{48} \approx 6048K$$

\subsubsection{Stefan Boltzmann Gesetz}
$$Waermestrom = \frac{P}{A} = \sigma * T^4$$
$\sigma$ ... Stefan-Boltzmann-Konstante = $5.7 * 10^{-8}\frac{W}{m^2 K^4}$
\\\\
Aus diesem Gesetz kann man ableiten, dass ein Faktor 2 bei der Temperatur einem Faktor 16 zur Abstrahlung entspricht.

\subsubsection{Bsp.: Glühbirnen}
Es soll bestimmt werden wie viel Watt eine Glühbirne bei einer Temperatur von 3000K abstrahlt.
Die Fläche des Wolfrahmdrahtes entspricht dabei ca. 0.5$cm^2$.

Die Werte können direkt in das zuvor erwähnte Gesetz eingesetzt werden.
$$\frac{P}{A} = \sigma * T^4$$
$$\frac{P}{5 * 10^{-5}} = \sigma * 3000^4$$
$$P = \sigma * 5 * 3000^4 * 10^{-5} \approx 46.2 * 10^{-8}W = 46.2nW$$

\section{Umwandlung von Wärmeenergie in mechanische Energie}
Wärmekraftmaschienen wandeln Wärmeenergie in mechanische Energie um.

\subsection{Cornot'scher Wirkungsgrad}
$$\eta_C = \frac{T_{Oben} - T_{Unten}}{T_{Oben}}$$

\section{Feuchte Luft}
\subsection{Absolute Luftfeuchtigkeit}
Ist die masse des in der Luft enthaltenen Wasserdampfes zum Luftvolumen $\frac{m}{V}$ in $\frac{g_{Wasser}}{kg_{Luft}}$.
\subsection{Relative Luftfeuchtigkeit}
Ist die Absolute Luftfeuchtigkeit dividiert durch die Maximale Luftfeuchtigkeit.
\subsection{Bsp.: Lufterwärmung}
Die relative Luftfeuchtigkeit in einem Raum mit $4m * 3m * 3m$ beträgt 40\% bei 24$^\circ$ Grad Celsius. Außerhalb beträgt die Temperatur 0$^\circ$ Grad und die Luftfeuchtigkeit beträgt 10\%.

Die halbe Luftmenge soll getauscht werden. Um mit der Berechnung fortzufahren benötigt man einige Werte aus dem Mollier-Diagramm\footnote{Zu finden unter: \href{https://de.wikipedia.org/wiki/Mollier-h-x-Diagramm}{https://de.wikipedia.org/wiki/Mollier-h-x-Diagramm}}.
Anhand des Linienverlaufs kann man erkennen, dass der Mischpunkt der beiden 13$^\circ$ Grad Celsius bei 45\% Luftfeuchtigkeit beträgt. Zusätzlich kann eine Absolute Luftfeuchtigkeit von 4$\frac{g}{kg}$ bestimmt werden.

Die Mischluft nun auf 20$^\circ$ Grad erhöht werden, was laut Diagramm zu eine Luftfeuchtigkeit von 25\% führt. Erwünscht wäre jedoch eine Luftfeuchtigkeit von 50\% und eine absolute Luftfeuchtigkeit von 7.5$\frac{g}{kg}$.
Dazu muss eine bestimmte Menge an Wasser zugeführt werden.

Die Masse des Wassers ist in der Formel zur absoluten Luftfeuchtigkeit enthalten. Das Beispiel ändert die absolute Luftfeuchtigkeit von 4 auf 7.5$\frac{g}{kg}$ also um 3.5$\frac{g}{kg}$. 
Das Volumen der Luft beträgt wie zu Beginn erwähnt $4m * 3m * 3m$ also $36m^3$. Nun kann in die Formel eingesetzt werden.
$$3.5\frac{g}{kg} = \frac{m g}{36m^3}$$

\section{Anomalie des Wassers}
\subsection{1. Anomalie}
Infolge der sperrigen Kristallstruktur dehnt sich Wasser beim erstarren aus und die Dichte sinkt.
\subsection{2. Anomalie}
Beim Abkühlen von Wasser setzt die Bildung der 6-Ecke aus, weshalb Wasser bei 4$^\circ$ Grad die höchste Dichte hat.

\section{Wirkungsgrad von Benzin und Diesel}
\begin{tabular}{l c c c c}
 & $T_{Arbeit}$ & $T_{Abgas}$ & $\eta_C$ & $\eta_{Real}$\\
 Otto (Benzin) & 2600K & 970K & 0.63 & 0.2 - 0.33\\
 Diesel & 2900K & 770K & 0.73 & 0.25 - 0.4
\end{tabular}


\end{document}